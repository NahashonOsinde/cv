\documentclass{examen}

\date{10 f�vrier 2007}
\formation{ENSMM}
\title{�preuve de \LaTeX}
\author{Guillaume Laurent}
\consignes{Dur�e : 5 min\\
Documents autoris�s : polycopi�s et manuscrits\\
Les parties sont totalement ind�pendantes.}


\begin{document}
\maketitle

Ce texte est avant une question.

\question{Cette question est en dehors d'un exercice}

Ce texte est entre deux questions.

\question{Celle-ci aussi}

Ce texte est apr�s une question.

% ---------------------------------------------------------------------
\exercice{Cet exercice est en dehors d'une partie}

Ce texte est avant une question.

\question{Cette question est dans un exercice}

Ce texte est entre deux questions.

\question{Celle-ci aussi}

Ce texte est apr�s une question.

% ---------------------------------------------------------------------
\exercice{Celui-ci aussi}

Ce texte est avant une question.

\question{Cette question est dans un exercice}

\question{Celle-ci aussi}

% =====================================================================
\partie{C'est une premi�re partie}

Ce texte est avant une question.

\question{Cette question est dans une partie et en dehors d'un exercice}

\question{Celle-ci aussi}

% ---------------------------------------------------------------------
\exercice{Cet exercice est dans une partie} 

Ce texte est avant une question.

\question{Cette question est dans un exercice et dans une partie}

\question{Celle-ci aussi}

% ---------------------------------------------------------------------
\exercice{Celui-ci aussi}

Ce texte est avant une question.

\question{Cette question est dans un exercice et dans une partie}

\question{Celle-ci aussi}

% =====================================================================
\partie{C'est une deuxi�me partie}

Ce texte est avant une question.

\question{Cette question est dans une partie et en dehors d'un exercice}

\question{Celle-ci aussi}

% ---------------------------------------------------------------------
\exercice{Cet exercice est dans une partie} 

Ce texte est avant une question.

\question{Cette question est dans un exercice et dans une partie}

\question{Celle-ci aussi}

% ---------------------------------------------------------------------
\exercice{Celui-ci aussi}

Ce texte est avant une question.

\question{Cette question est dans un exercice et dans une partie}

\question{Celle-ci aussi}

\end{document}
